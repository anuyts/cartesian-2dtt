\documentclass[11pt]{article}
\usepackage{times}

\usepackage{fullpage}
\usepackage{ifthen}
\usepackage{latexsym,amsmath,amsfonts,amssymb,stmaryrd,amsthm}
\usepackage{verbatim,fancyvrb}
\usepackage{multicol}
\usepackage{proof}
\usepackage[square]{natbib}

\usepackage[backgroundcolor=white,bordercolor=red]{todonotes}
\usepackage{xypic}

\usepackage{ucs}
\usepackage[utf8x]{inputenc}
\usepackage[T1]{fontenc}


\newcommand{\todoi}[1]{\todo[inline]{#1}}


% braces
\newcommand{\set}[2]{\left\{#1\,\middle|\,#2\right\}}
\newcommand{\paren}[1]{\left(#1\right)}
\newcommand{\brac}[1]{\left[#1\right]}
\newcommand{\accol}[1]{\left\{#1\right\}}
\newcommand{\angles}[1]{\left\langle#1\right\rangle}
\newcommand{\interp}[1]{\left\llbracket#1\right\rrbracket}
\newcommand{\norm}[1]{\left\lVert#1\right\rVert}

\newcommand{\vfi}{\varphi}

\newcommand{\substdup}[3]{\ensuremath{(#1_{#2};#3)}}
\newcommand{\substtw}[4]{\ensuremath{[#1 \xrightarrow{#3} #2] / [#4]}}

\newcommand{\idtp}[3]{#2 =_{#1} #3}

\newcommand{\vcond}[0]{\ensuremath{{\mathord{*}}}}
\newcommand{\vconabs}[2]{\ensuremath{#1.#2}}

\newcommand{\togglev}[2]{\ensuremath{#1^{#2}}}

\newcommand{\dupctx}[2]{\brac{#1}^{#2}}

\newcommand{\picat}[3]{\ensuremath{\textnormal{\dsd{Lim}}_{\tptmns{#1}{#2}} \,\dcd{#3}}}
\newcommand{\sigmacatstar}[4]{\ensuremath{\textnormal{$\Sigma$}^{#1}\,\tptmns{#2}{#3}.\,\dcd{#4}}}
\newcommand{\sigmacat}[3]{\ensuremath{\textnormal{$\Sigma$}\,(\tptmns{#1}{#2}).\,\dcd{#3}}}
\newcommand{\sigmatwcat}[3]{\ensuremath{\textnormal{$\widetilde\Sigma$}\,(\tptmns{#1}{#2}).\,\dcd{#3}}}

\newcommand{\piset}[3]{\ensuremath{\textnormal{$\Pi$}^{\dsd{s}}\,\tptmns{#1}{#2}.\,\dcd{#3}}}
\newcommand{\sigmaset}[3]{\ensuremath{\textnormal{$\Sigma$}^{\dsd{s}}\,(\tptmns{#1}{#2}).\,\dcd{#3}}}
%% \newcommand{\pitp}[4]{\ensuremath{\textnormal{$\Pi$}^{#3}\,\tptmns{#1}{#2}.\,\dcd{#4}}}
\newcommand{\sigmatp}[3]{\ensuremath{\textnormal{$\Sigma$}^{\dsd{t}}\,\tptmns{#1}{#2}.\,\dcd{#3}}}

\newcommand{\pico}[3]{\ensuremath{\textnormal{$\Pi$}^\coc(\tptmns{#1}{#2}).\,\dcd{#3}}}
\newcommand{\laco}[3]{\ensuremath{\lambda(\tpvar{#1}{\coc}{#2}).#3}}
\newcommand{\picon}[3]{\ensuremath{\textnormal{$\Pi$}^\conc(\tptmns{#1}{#2}).\,\dcd{#3}}}
\newcommand{\lacon}[3]{\ensuremath{\lambda(\tpvar{#1}{\conc}{#2}).#3}}

\newcommand{\homset}[3]{\ensuremath{\dsd{hom}_{#1}(#2,#3)}}
\newcommand{\homtp}[3]{\ensuremath{\dsd{Hom}_{#1}(#2,#3)}}

\newcommand{\bool}[0]{\dsd{2}}

\newcommand\Elp[1]{\dsd{El}\ep(#1)}
\newcommand\Eln[1]{\dsd{El}\en(#1)}
\newcommand\Elpn[1]{\dsd{El}\epn(#1)}

\newcommand\Set[0]{\ensuremath{\mathbf{Set}}}

\newcommand\wfdtp[1]{\ensuremath{#1 \, \dsd{cat}}}
\newcommand\wfctx[1]{\ensuremath{#1 \, \dsd{ctx}}}

%% \newcommand\tctx[2]{\ensuremath{#1 \mid #2}}

\newcommand\wftp[2]{\ensuremath{#1 \vdash #2 \, \dsd{type}}}

%% \newcommand\voftp[3]{\ensuremath{#1 \vdash #2 \, {{:}^{#3}} \, \dcd{#4}}}

\newcommand\withop[1]{\ensuremath{(#1)^{\dsd{op}} \times #1}}

\newcommand{\oftp}[3]{\ensuremath{{#1} \, \vdash \dcd{#2} \, \dcd{:} \, #3}}
\newcommand{\ofdtp}[3]{\ensuremath{{#1} \, \vdash \dcd{#2} \, \dcd{:} \, #3}}
\newcommand{\oftpe}[4]{\ensuremath{{#1} \, \vdash \dcd{#2} \, \dcd{:}^{#3} \, #4}}

\newcommand\admiss[0]{\textbf{admiss}}
\newcommand\deriv[0]{\textbf{deriv}}

\newcommand\fliptm[2]{\ensuremath{\dsd{flip}^{#1} \, #2}}

\newcommand\deq[0]{\ensuremath{\equiv}}

\newcommand\vvoftp[5]{\ensuremath{#1 \vdash #2 \, {\,^{#3}{:}^{#4}} \, \dcd{#5}}}
\newcommand\vvdeqtm[6]{#1 \vdash #2 \deq #3 \, {\,^{#4}{:}^{#5}} \, #6}
\newcommand\vvdeqtmnc[5]{#1 \deq #2 \, {\,^{#3}{:}^{#4}} \, #5}
\newcommand\F[2]{\ensuremath{\mathsf{F}_{#1} #2}}
\newcommand\wfsub[3]{\ensuremath{#1 \vdash #2 : #3}}

\newcommand\tsubst[2]{\ensuremath{#1 [ #2 ]}}
\newcommand\idsubst[0]{\ensuremath{\dsd{idsubst}}}

\newcommand{\ep}{\ensuremath{^{\text{\tiny +}}}}
\newcommand{\en}{\ensuremath{^{\text{\small -}}}}
\newcommand{\epc}{\ensuremath{^{\text{\tiny {$\oplus$}}}}}
\newcommand{\enc}{\ensuremath{^{\text{\tiny {$\ominus$}}}}}
\newcommand{\epn}{\ensuremath{^{\text{\tiny $\pm$}}}}

\newcommand{\forw}[1]{\ensuremath{\overrightarrow{#1}}}
\newcommand{\backw}[1]{\ensuremath{\overleftarrow{#1}}}

\newcommand\co{\ensuremath{{\text{\tiny $\mathord{+}$}}}}
\newcommand\con{\ensuremath{{\text{\tiny $\mathord{-}$}}}}

\newcommand\coc{\ensuremath{{\text{\tiny $\mathord{\oplus}$}}}}
\newcommand\conc{\ensuremath{{\text{\tiny $\mathord{\ominus}$}}}}


\newcommand\inv{\ensuremath{{\text{\tiny $\mathord{\pm}$}}}}
\newcommand\flip[1]{\ensuremath{\overline{#1}}}
\newcommand\ectx[5]{\ensuremath{#1 , #2 \,^{#3}{:}^{#4} #5}}

\newcommand\uncoresubst[0]{\dsd{uncore}}

\newcommand\letfunc[4]{\ensuremath{\dsd{letfunc}^{#1}_{#2}(#3,#4)}}

\newcommand\letcore[3]{\ensuremath{\dsd{letcore}_{#1}(#2,#3)}}

\newcommand\tc[2]{\ensuremath{#1 \Rightarrow #2}}
\newcommand\comph[2]{\ensuremath{#1 \mathbin{\circ_2} #2}}

\newcommand\uncorep[0]{\ensuremath{\dsd{uncore}^\co}}
\newcommand\uncoren[0]{\ensuremath{\dsd{uncore}^\con}}

%\newcommand\vtptm[3]{\ensuremath{\dcd{#1} \,^{#2}{:}^{\,} \dcd{#3}}}
%\newcommand\tptmv[3]{\ensuremath{\dcd{#1} \,^{\,}{:}^{#2} \dcd{#3}}}
\newcommand\catvar[3]{\ensuremath{\dcd{#1} \,^{#2}{:}^{\,} \dcd{#3}}}
\newcommand\twcatvar[3]{\ensuremath{[\dcd{#1}] \,^{#2}{:}^{\,} \dcd{#3}}}
\newcommand\tpvar[3]{\ensuremath{\dcd{#1} \,^{\,}{:}^{#2} \dcd{#3}}}

\newcommand\op[1]{\ensuremath{#1 ^{\dsd{op}}}}

\newcommand\coretp[1]{\ensuremath{{\dsd{Core}} \: #1}}

\newcommand\Ob[1]{\ensuremath{Ob(#1)}}
\newcommand\morphismtp[3]{\ensuremath{#2 \longrightarrow_{#1} #3}}

\newcommand\inF[3]{\ensuremath{\dsd{in}^{#1}_{#2}(#3)}}
\newcommand\outF[3]{\ensuremath{\dsd{out}^{#1}_{#2}(#3)}}

\newcommand\optp[1]{\ensuremath{{#1^{\dsd{op}}}}}
\newcommand\optm[1]{\ensuremath{\dsd{op}(#1)}}
\newcommand\unop[1]{\ensuremath{\dsd{unop}(#1)}}

\newcommand\dagtp[1]{\ensuremath{{#1^{\dagger}}}}
\newcommand\dagtm[1]{\ensuremath{\dsd{dag}(#1)}}
\newcommand\undag[1]{\ensuremath{\dsd{undag}(#1)}}

\newcommand\obtp[1]{\ensuremath{{\dsd{Ob}(#1)}}}
\newcommand\obtm[1]{\ensuremath{\dsd{ob}(#1)}}
\newcommand\unob[1]{\ensuremath{\dsd{unob}(#1)}}
\newcommand\unobv[2]{\ensuremath{\dsd{unob}_{#1}(#2)}}

\newcommand\disctp[1]{\ensuremath{{\dsd{Discr}(#1)}}}
\newcommand\disctm[1]{\ensuremath{\dsd{discr}(#1)}}
\newcommand\letdisc[3]{\ensuremath{\dsd{letdisc}_{#1}(#2,#3)}}

\newcommand\coretm[1]{\ensuremath{\dsd{core}(#1)}}
\newcommand\letc[3]{\ensuremath{\dsd{letc}_{#1}(#2,#3)}}

%% ----------------------------------------------------------------------


\newcommand{\Hom}[3]{\ensuremath{\dsd{Hom}_{#1}(#2,#3)}}
\newcommand{\Hombind}[4]{\ensuremath{\dsd{Hom}_{#1}^{#2}(#3,#4)}}

\newcommand{\Eq}[3]{\ensuremath{\dsd{Eq}_{#1}(#2,#3)}}

\newcommand\ford{\ensuremath{{\text{\tiny +}}}}
\newcommand\backd{\ensuremath{{\text{\small -}}}}
\newcommand\symd{\ensuremath{{\text{\tiny $\pm$}}}}
\newcommand\invd[1]{\ensuremath{\overline{#1}}}
\newcommand\anyctx[5]{\ensuremath{#1 \stackrel{#2}{,} #3 \stackrel{#5}{:} #4}} 

\newcommand\coctx[4]{\ensuremath{#1 \stackrel{\dsd{co}}{,} #2 {:} #3 [#4]}} 
\newcommand\conctx[4]{\ensuremath{#1 \stackrel{\dsd{con}}{,} #2 {:} #3 [#4]}} 

\newcommand\woftp[4]{\ensuremath{#1 \vdash \dcd{#2} \stackrel{#4}{:} \dcd{#3}}}

\newcommand\ctx[1]{\text{\ensuremath{#1}}}

\newcommand\FIXME[1]{\todoi{#1}}
%\newcommand\FIXME[1]{\text{\textbf{FIXME:} #1}}

\newcommand\wtp[2]{\ensuremath{#1 ^ #2}}

\newcommand\I{\dsd{{\mathbb{I}}}}
\newcommand\Iz{\dsd{zero}}
\newcommand\Io{\dsd{one}}
\newcommand\Iseg[1]{\dsd{seg}_{#1}}
\newcommand\Ielim[5]{\ensuremath{\dsd{\I\text{-}elim}_{#1}(#2,#3,#4,#5)}}

\newcommand\transportfor[3]{\dsd{transport}_{#1} \: #2 \: #3}

%% \newcommand\definit[0]{\: \mathbin{\text{iff}} \: }

%% \newcommand\co[1]{\ensuremath{\dcd{#1}\ep}}
%% \newcommand\contra[1]{\ensuremath{\dcd{#1}\en}}
%% \newcommand\cocon[1]{\ensuremath{#1 \epn}}
%% \newcommand\conco[1]{\ensuremath{#1 ^\mp}}

%% \newcommand\cons[3]{\ensuremath{#1 \mathbin{,} \tptmns{#2}{#3}}}
%% % \newcommand\wfotp[2]{\ensuremath{#1\,\vdash \dcd{#2}\:\cocon {\dsd {type}}}}
%% \newcommand\consp[3]{\cons{#1}{#2}{\co #3}}
%% \newcommand\consn[3]{\cons{#1}{#2}{\contra #3}}
%% \newcommand\gctxr[0]{\dsd{Ctx^\leftarrow}}
%% \newcommand\gctxs[0]{\dsd{Ctx}}
%% \newcommand\gsort[0]{\dsd{GSort}}
%% \newcommand\ctx[1]{\text{\ensuremath{#1}}}
%% \newcommand\data[2]{\ensuremath{\langle \ctx{#1} \rangle \dcd{#2}}}
\newcommand\ite[4]{\ensuremath{\dsd{if}_{#1}(\dcd{#2},\dcd{#3},\dcd{#4})}}
%% \newcommand\tsubst[2]{\ensuremath{\dcd{#1} [ #2 ]}}


%% \renewcommand\abort[1]{\ensuremath{\dsd{abort} \, \dcd{#1}}}
%% \newcommand\wfsub[3]{\ensuremath{#1 \vdash #2 : #3}}
%% \newcommand\homj[3]{\ensuremath{#2 \Longrightarrow_{#1} #3}}
%% \newcommand\ofhomj[4]{\ensuremath{#1 : \homj{#2}{#3}{#4}}}
%% \newcommand\ofhom[5]{\ensuremath{#1 \vdash #2 : \homj{#3}{#4}{#5}}}
%% % \newcommand\ofhominj[4]{#3 \mapsto^{#2}_{#1} {#4}}
%% % \newcommand\ofhomin[6]{\ensuremath{#1 \vdash #2 : #5 \mapsto^{#4}_{#3} {#6}}}
%% \newcommand\set[0]{\dsd{set}}

%% \newcommand\el[1]{\ensuremath{El(#1)}}

%% \newcommand\map[3]{\ensuremath{\dsd{map}_{#1} \ #2 \ #3}}
%% \newcommand\maps[3]{\ensuremath{\dsd{map}^1_{#1} \ #2 \ #3}}
%% \renewcommand\id[3]{\ensuremath{\dsd{Id}_{#1} \ #2 \ #3}}
%% \renewcommand\subst[3]{\ensuremath{#1 [ #2 / #3]}}
%% \newcommand\hlam[2]{\dcd{#1.{#2}}}
%% \newcommand\ext[5]{\ensuremath{\lambda_{#1} \dcd{#2,#3,#4.#5}}}
%% \newcommand\unext[4]{\dcd{#1 \ #2 \ #3 \ #4}}
%% \newcommand\comp[2]{\ensuremath{#1 \circ #2}}
%% \newcommand\resp[2]{\ensuremath{\dsd{resp} \ {#1} \ #2}}

%% \newcommand\ido[1]{\ensuremath{\ids_{#1}}}
%% \newcommand\idt[2]{\ensuremath{\ids^{#1}_{#2}}}

%% \newcommand{\deq}[3]{\ensuremath{#1\,\vdash\,\tp{#2} \, \equiv \, \tp{#3}}}
%% \newcommand{\deqctx}[2]{\ensuremath{#1 \equiv #2}}

%% \newcommand\wffunc[3]{\ensuremath{#1 : \functp {#2} {#3}}}
%% \newcommand\nttp[2]{\ensuremath{#1 \Longrightarrow #2}}
%% \newcommand\wfnt[5]{\ensuremath{#1 : \nttp {#2} {#3} : \functp {#4} {#5}}}

\newcommand\grothp[2]{\ensuremath{\int^{\dsd{co}}_{#1} #2}}
\newcommand\groth[3]{\ensuremath{\int^{#1}_{#2} #3}}
\newcommand\grothn[2]{\ensuremath{\int^{\dsd{contra}}_{#1} #2}}

\newcommand\compo[2]{\ensuremath{#1 \cdot #2}}

%% \newcommand\sty[1]{\ensuremath{\dsd {Ty} \ #1}}
%% \newcommand\stm[2]{\ensuremath{\dsd {Tm} \ #1 \ #2}}

%% \newcommand\dtt{2DTT}
%% \newcommand\cats{\ensuremath{Cat}}

%% \newcommand\var[1]{\ensuremath{\dsd{var}(#1)}}
%% \newcommand\propos[1]{\ensuremath{\dsd{prop}(#1)}}
%% \newcommand\nd[2]{\ensuremath{\dsd{nd} \, #1 \, #2}}

%% \newcommand\iin[1]{\dsd{in} \, #1}
%% \newcommand\iout[1]{\dsd{out} \, #1}

%% \newcommand\sets[0]{\dcd{Sets}}
%% \newcommand\idata[2]{\app{\dsd{#1}}{#2}}

%% \newcommand\realid[1]{\dsd{id}_{#1}^{#1}}


\newcommand\refl[1]{\ensuremath{\dsd{refl}_{#1}}}
\newcommand\refls[0]{\dsd{refl}}

\newcommand\idm[1]{\ensuremath{\dsd{id}_{#1}}}
\newcommand\morind[3]{\ensuremath{\dsd{hind}_{#1}(#2,#3)}}
\newcommand\morindco[3]{\ensuremath{\dsd{hind}^\co_{#1}(#2,#3)}}
\newcommand\morindcon[3]{\ensuremath{\dsd{hind}^\con_{#1}(#2,#3)}}

\newcommand\splitsig[3]{\ensuremath{\dsd{split}_{#1}(#2,#3)}}

%% \newcommand{\idi}[1]{\dsd{idi} \: #1}
%% \newcommand{\ide}[1]{\dsd{ide} \: #1}

%% \newcommand\mtext[1]{\text{\emph{#1}} \\ \\}
%% \renewcommand\irl[1]{\text{\emph{#1}}}

\newcommand\Core[1]{\ensuremath{{#1^\dsd{core}}}}
\newcommand\Cat{\ensuremath{\bold{Cat}}}
\newcommand\Gpd{\ensuremath{\bold{Gpd}}}

\newcommand\II{\ensuremath{\mathbb{I}}}

\newcommand\vdimtm[2]{\ensuremath{\vtptm{#1}{#2}{\,}{\dsd{dim}}}}
\newcommand\vwfdim[3]{\ensuremath{\vvoftp{#1}{#2}{#3}{\,}{\dsd{dim}}}}
\newcommand\wfdim[2]{\ensuremath{\vvoftp{#1}{#2}{}{\,}{\dsd{dim}}}}

\newcommand\rev[1]{\ensuremath{1-#1}}

\newcommand\nsubst[3]{\ensuremath{#1 \langle #2 / #3 \rangle}}
\newcommand\subst[3]{\ensuremath{#1 [ #2 / #3 ]}}

\newcommand\wfctxs[1]{\ensuremath{#1 \: \dsd{ctx}_{\Sm}}}
\newcommand\wfctxc[2]{\ensuremath{#1 \vdash #2 \, \dsd{ctx}_{\Cm}}}

%% \newcommand\wftps[2]{\ensuremath{#1 \vdash #2 \, \dsd{type}_{\Sm}}}
\newcommand\wftpsf[3]{\ensuremath{#1 ; #2 \vdash #3 \,{:}\, \dsd{type}_{\Sm}}} %% fibration of sets
\newcommand\wftpp[3]{\ensuremath{#1 ; #2 \vdash #3 \,{:}\, \dsd{type}_{p}}}
\newcommand\wftpc[3]{\ensuremath{#1 ; #2 \vdash #3 \, \dsd{type}_{\Cm}}}
\newcommand\wftpcc[2]{\ensuremath{#1 \vdash #2 \, \dsd{type}_{\Cm}}} %% constant

%\newcommand\types{\ensuremath{\dsd{set}\/}}
\newcommand\types{\ensuremath{\mathbb{S}\dsd{et}}}
\newcommand{\Twisted}[1]{\ensuremath{\mathbb{T}\dsd{w}\paren{#1}}}

\newcommand\vwfdimc[4]{\ensuremath{\vvoftp{#1;#2}{#3}{#4}{\,}{\dsd{dim}}}}
\newcommand\wfdimc[3]{\ensuremath{\vvoftp{#1;#2}{#3}{}{\,}{\dsd{dim}}}}

\newcommand\oftps[3]{\ensuremath{#1 \vdash #2 \, {:} \, \dcd{#3}}}

\newcommand\voftpc[5]{\ensuremath{#1 ; #2 \vdash #3 \, {{:}^{#4}} \, \dcd{#5}}}
\newcommand\vvoftpc[6]{\ensuremath{#1 ; #2 \vdash #3 \, {\,^{#4}{:}^{#5}} \, \dcd{#6}}}

\newcommand\Cm{\dsd{c}}
\newcommand\Sm{\dsd{s}}

\newcommand\C{\ensuremath{\mathbb{C}}}
\newcommand\D{\ensuremath{\mathbb{D}}}
\newcommand\G{\ensuremath{\Gamma}}

\newcommand\vptptm[3]{\ensuremath{\dcd{#1} \,^{#2}{:}^{\,} \dcd{#3}}}
\newcommand\vstptm[3]{\ensuremath{\dcd{#1} \,^{\,}{:}^{#2} \dcd{#3}}}
\newcommand\oftpc[4]{\ensuremath{#1 ; #2 \vdash #3 \, {{:}} \, \dcd{#4}}}
\newcommand\oftpsf[4]{\ensuremath{#1 ; #2 \vdash #3 \, {:} \, \dcd{#4}}}
\newcommand\oftpp[4]{\ensuremath{#1 ; #2 \vdash #3 \, {{:}} \, \dcd{#4}}}

\newcommand\promote[3]{\dsd{promote}_{#1}(#2,#3)}

\newcommand\nlam[2]{\ensuremath{\langle #1 \rangle #2}}
\newcommand\napp[2]{\ensuremath{#1 @ #2}}

\newcommand{\HomO}[3]{\ensuremath{\dsd{IHom}_{#1}(#2,#3)}}

\newcommand{\dcd}[1]{\ensuremath{\mathit{#1}}}
\newcommand{\dsd}[1]{\ensuremath{\mathsf{#1}}}
\newcommand{\tptm}[2]{\ensuremath{#1 \, \dcd{:} \, #2}}
\newcommand{\tptmns}[2]{\ensuremath{#1 \dcd{:} #2}}

\newcommand{\larr}[3]{\ensuremath{\dcd{#2}\rightarrow_{\dcd{#1}} \dcd{#3}}}
\newcommand{\arr}[2]{\larr{}{#1}{#2}}
\newcommand{\picl}[3]{\ensuremath{\textnormal{$\Pi$}\,\tptmns{#1}{#2}.\,\dcd{#3}}}
\newcommand{\sigmacl}[3]{\ensuremath{\textnormal{$\Sigma$}\,\tptmns{#1}{#2}.\,\dcd{#3}}}

\newcommand{\lprd}[3]{\ensuremath{\dcd{#2} \times_{\dcd{#1}} \dcd{#3}}}
\newcommand{\lsm}[3]{\ensuremath{\dcd{#2} +_{\dcd{#1}} \dcd{#3}}}
\newcommand{\prd}[2]{\lprd{}{#1}{#2}}
\newcommand{\sm}[2]{\lsm{}{#1}{#2}}

\newcommand{\ulam}[2]{{\lambda\,#1.\,#2}}
\newcommand{\lam}[3]{{\lambda\,\tptmns{#1}{#2}.\,#3}}
\newcommand{\app}[2]{\dcd{#1 \: #2}}
\newcommand{\appp}[3]{\dcd{#1 \: #2 \: #3}}
\newcommand{\apppp}[4]{\dcd{#1 \: #2 \: #3 \: #4}}
\newcommand{\appppp}[5]{\dcd{#1 \: #2 \: #3 \: #4 \: #5}}
\newcommand{\apppppp}[6]{\dcd{#1 \: #2 \: #3 \: #4 \: #5 \: #6}}

% put parens around each arg, associating to the left
\newcommand{\lpapp}[2]{\dcd{#1\,(#2)}}
\newcommand{\lpappp}[3]{\dcd{#1(#2)(#3)}}
\newcommand{\lpapppp}[4]{\dcd{#1\,(#2)\,(#3)\,(#4)}}
\newcommand{\lpappppp}[5]{\dcd{#1\,(#2)\,(#3)\,(#4)\,(#5)}}
\newcommand{\lpapppppp}[6]{\dcd{#1\,(#2)\,(#3)\,(#4)\,(#5)\,(#6)}}

% put parens around args, associating to the right 
\newcommand{\rpapp}[2]{\dcd{#1\,(#2)}}
\newcommand{\rpappp}[3]{\dcd{#1\,(#2\,(#3))}}
\newcommand{\rpapppp}[4]{\dcd{#1\,(#2\,(#3\,(#4)})}
\newcommand{\rpappppp}[5]{\dcd{#1\,(#2\,(#3\,(#4\,(#5))))}}
\newcommand{\rpapppppp}[6]{\dcd{#1\,(#2\,(#3\,(#4\,(#5\,(#6)))))}}

%%%%%%%%%%%%%%%%%%%%%%%%%%%%%%%%%%%%%%%%%%%%%%%%%%%%%%%%%%%%%%%%%%%%%%%%
% Lambdas

% a capital Lambda
\newcommand{\lLam}[4]{\dcd{\Lambda_{#1}\,\kdcnns{#2}{#3}.\,#4}}
\newcommand{\Lam}[3]{\lLam{}{#1}{#2}{#3}}
\newcommand{\uLam}[2]{\dcd{\Lambda\,\cn{#1}.\,#2}}

% juxtaposition with brackets 1[2]
\newcommand{\App}[2]{\lApp{}{#1}{#2}}
\newcommand{\Appp}[3]{\lAppp{}{#1}{#2}{#3}}
\newcommand{\Apppp}[4]{\lApppp{}{#1}{#2}{#3}{#4}}
\newcommand{\Appppp}[5]{\lAppppp{}{#1}{#2}{#3}{#4}{#5}}
\newcommand{\Apppppp}[6]{\lApppppp{}{#1}{#2}{#3}{#4}{#5}{#6}}

% bracket application with a level marker
\newcommand{\lApp}[3]{\dcd{#2[#3]_{#1}}}
\newcommand{\lAppp}[4]{\dcd{#2[#3][#4]_{#1}}}
\newcommand{\lApppp}[5]{\dcd{#2[#3][#4][#5]_{#1}}}
\newcommand{\lAppppp}[6]{\dcd{#2[#3][#4][#5][#6]_{#1}}}
\newcommand{\lApppppp}[7]{\dcd{#2[#3][#4][#5][#6][#7]_{#1}}}

% with a level marker
\newcommand{\lpair}[3]{\ensuremath{\dcd{(#2,#3)_{#1}}}}
\newcommand{\lfst}[2]{\app{\dsd{fst}_{#1}}{#2}}
\newcommand{\lsnd}[2]{\app{\dsd{snd}_{#1}}{#2}}

\newcommand{\lemptytuple}[1]{\dcd{()_{#1}}}

%% without
\newcommand{\pair}[2]{\lpair{}{#1}{#2}}
\newcommand{\fst}[1]{\lfst{}{#1}}
\newcommand{\snd}[1]{\lsnd{}{#1}}

\newcommand{\emptytuple}[0]{\lemptytuple{}}

\newcommand{\linl}[3]{\app{\dsd{inl}_{#1}[#2]}{#3}}
\newcommand{\linr}[3]{\app{\dsd{inr}_{#1}[#2]}{#3}}
\newcommand{\lcase}[6]{\ensuremath{\dsd{case}_{\dcd{#1}}\:\dcd{#2}\:\dcd{of}\:
  (\app{\dsd{inl}}{#3}\,\dcd{\Rightarrow}\,\dcd{#4}\:\dcd{|}\:\app{\dsd{inr}}{#5}\,\dcd{\Rightarrow}\,\dcd{#6} )}}

\newcommand{\sumscase}[5]{\lcase{}{#1}{#2}{#3}{#4}{#5}}

\newcommand{\labort}[3]{\app{\dsd{abort}_{#1}[#2]}{#3}}

\newcommand{\inl}[2]{\linl{}{#1}{#2}}
\newcommand{\inr}[2]{\linr{}{#1}{#2}}


\newcommand{\abort}[2]{\labort{}{#1}{#2}}

% gamma proves t has type tau judgement
%% \newcommand{\oftp}[3]{\ensuremath{#1 \, \vdash \dcd{#2} \, \dcd{:} \, \cn{#3}}}
% well-formed type
%% \newcommand{\wftp}[2]{\ensuremath{#1\,\vdash \dcd{#2}\:\dsd{type}}}
\newcommand{\deqtm}[4]{\ensuremath{#1\,\vdash\,\tm{#2} \, \equiv \, \tm{#3} \, \dcd{:} \, \tp{#4}}}
%% \newcommand{\wfctx}[1]{\ensuremath{#1 \: \dsd{ctx}}}

\renewcommand{\cite}[1]{\citep{#1}}

%% small tightcode, with space around it
\newenvironment{stcode}
{\smallskip
\begin{small}
\begin{tightcode}}
{\end{tightcode}
\end{small}
\smallskip}

\DeclareUnicodeCharacter{9001}{$\langle$} %% 〈
\DeclareUnicodeCharacter{9002}{$\rangle$} %%  〉
\DeclareUnicodeCharacter{12314}{$\llbracket$} %% 〚
\DeclareUnicodeCharacter{12315}{$\rrbracket$}  %% 〛
\DeclareUnicodeCharacter{8872}{$\vDash$} %% ⊢ 
\DeclareUnicodeCharacter{9656}{\!\!\!}  %% ▸
\DeclareUnicodeCharacter{9657}{$\triangleright$} %% ▹
\DeclareUnicodeCharacter{8594}{$\shortrightarrow$} %% →
\DeclareUnicodeCharacter{8675}{$\downshift$} %% ⇣
\DeclareUnicodeCharacter{8593}{$\upshift$} %% ↑
\DeclareUnicodeCharacter{9711}{$\bigcirc$} %% ◯
\DeclareUnicodeCharacter{8667}{$^\shortrightarrow$} %% ⇛
\DeclareUnicodeCharacter{8596}{$^\leftrightarrow$} %% ↔
\DeclareUnicodeCharacter{10224}{$\uparrow$} %% ⟰
\DeclareUnicodeCharacter{10225}{$\downarrow$} %% ⟱

\newcommand\textPsi{\ensuremath{\Psi}}
\newcommand\textXi{$\Xi$}
\newcommand\textDelta{$\Delta$}
\newcommand\textGamma{$\Gamma$}
\newcommand\textsigma{$\sigma$}
\newcommand\textSigma{$\Sigma$}
\newcommand\textTheta{$\Theta$}
\newcommand\texttheta{$\theta$}
\newcommand\textPi{$\Pi$}
\newcommand\textrho{$\rho$}
\newcommand\textphi{$\varphi$}
\newcommand\textlambda{$\lambda$}
\newcommand\texttau{$\tau$}
\newcommand\texteta{$\eta$}
%% FIXME renewcommand when using hyperref
\newcommand\textmu{$\mu$}
\newcommand\textalpha{$\alpha$}
\newcommand\textepsilon{$\epsilon$}
\newcommand\textkappa{$\kappa$}
\newcommand\textOmega{$\Omega$}
\newcommand\textmho{$\mho$}
\newcommand\textgamma{$\gamma$}
\newcommand\textpi{$\varpi$}
\newcommand\textpsi{\ensuremath{\psi}}


\DefineVerbatimEnvironment{code}{Verbatim}{fontsize=\small,fontfamily=tt}

\theoremstyle{plain}
\newtheorem{thm}{Theorem}[section]
\newtheorem{lem}{Lemma}[section]

\title{All's Well That Ends Well}
\author{Daniel R. Licata \and Andreas Nuyts \and Patrick Schultz \and Michael Shulman}

\begin{document}
\maketitle

\section{Judgements}

\begin{itemize}

\item \wfdtp{\C} means \C\/ is a category; note that categories cannot
  depend on the context at all.  

\item \wfctx{\G} means $\Gamma$ is a category

\item \ofdtp{\G}{c}{\C} means $c$ is a functor
  ${\G} \to \C$, we also write $S$ when $c$ is of category $\types$.  

\item \oftp{\G}{t}{S} where
  \ofdtp{\Gamma}{S}{\types} means $t$ is an element of $\varprojlim S$, or equivalently a section of $\pi_1 : \int_\Gamma^+ T \to \Gamma$.

\end{itemize}

\section{Contexts and Structural Rules}

We write $x$ for a variable. 

\[
\begin{array}{l}
v ::= \co \mid \con \\
w ::= \coc \mid \conc \\
\end{array}
\]

\[
\begin{array}{c}
\infer{\wfctx{\cdot}}{}
\qquad
\infer{\wfctx{(\Gamma,\catvar{x}{v}{\C})}}
      {\wfctx{\Gamma} & 
       \wfdtp{\C}
      } 
\qquad
\infer{\wfctx{(\Gamma,\twcatvar{x}{v}{\C})}}
      {\wfctx{\Gamma} & 
       \wfdtp{\C}
      } 
\qquad
\infer{\wfctx{(\Gamma,\tpvar{x}{w}{S})}}
      {\wfctx{\Gamma} & 
        \wftp{\Gamma}{S}
      } 
\end{array}
\]
\todoi{Or do we want to see $\twcatvar x v{\C}$ as an abbreviation for $\catvar{x^*}{\flip v}{\C}, \catvar{x}{v}{\C}, \tpvar{\idm x}{\coc}{\Hom{\C}{x^*}{x}}$?}

\[
\begin{array}{rcl}
\Gamma^\co & := & \Gamma \\
\cdot^\con & := & \cdot \\
({\Gamma},\catvar{x}{v}{\C})^\con & := & {\Gamma^\con},\catvar{x}{\flip{v}}{\C} \\
({\Gamma},\twcatvar{x}{v}{\C})^\con & := & {\Gamma^\con},\twcatvar{x}{\flip{v}}{\C} \\
({\Gamma},\tpvar{x}{w}{S})^\con & := & {\Gamma^\con},\tpvar{x}{\flip w}{S} \\
\end{array}
\]

\[
\infer{\catvar{x}{v}{\C} \in (\Gamma,\catvar{x}{v}{\C})}
      {}
\qquad
\infer{\catvar{x^*}{\flip v}{\C} \in (\Gamma,\twcatvar{x}{v}{\C})}
      {}
\qquad
\infer{\catvar{x}{v}{\C} \in (\Gamma,\twcatvar{x}{v}{\C})}
      {}
\qquad
\infer{\tpvar{\idm x}{(v)}{\C} \in (\Gamma,\twcatvar{x}{v}{\C})}
      {}
\]

\[
\infer{\catvar{x}{v}{\C} \in (\Gamma,\_)}
      {\catvar{x}{v}{\C} \in \Gamma}
\qquad
\infer{\tpvar{t}{w}{S} \in (\Gamma,\tpvar{x}{w}{S})}
      {}
\qquad
\infer{\tpvar{t}{w}{S} \in (\Gamma,\_)}
      {\tpvar{t}{w}{S} \in \Gamma}
\]

\[
\infer{\ofdtp{\Gamma}{x}{\C}}{\catvar{x}{\co}{\C} \in \Gamma}
\qquad
\infer{\oftp{\Gamma}{t}{S}}
      {\tpvar{t}{\coc}{S} \in \Gamma}
\]

\[
\infer{\wfsub{\Gamma}{\cdot}{\cdot}}{} 
\qquad
\infer{\wfsub{\Gamma}{(\theta,c/x)}{(\Delta,\catvar{x}{v}{\C})}}
      {\wfsub{\Gamma}{\theta}{\Delta} & 
        \ofdtp{\Gamma^v}{c}{\C}
      }
\qquad
\infer{\wfsub{\Gamma}{(\theta,\substtw{c'}{c}{\vfi}{x})}{(\Delta,\twcatvar{x}{v}{\C})}}
      {\wfsub{\Gamma}{\theta}{\Delta} & 
        \ofdtp{\Gamma^v}{c}{\C} &
        \ofdtp{\Gamma^{\flip v}}{c'}{\C} &
        \oftp{\Gamma^v}{\vfi}{\Hom{\C}{c'}{c}}
      }
%% \infer[\textbf{admiss}]
%%       {\Gamma,\subst{\Gamma'}{c}{x} \vdash \subst{J}{c}{x}}
%%       {\Gamma,\catvar{x}{v}{\C},\Gamma' \vdash {J} &
%%         \oftp{\Gamma^v}{c}{\C}}
%% \qquad
%% \infer[\textbf{admiss}]
%%       {\Gamma,\subst{\Gamma'}{t}{x} \vdash \subst{J}{t}{x}}
%%       {\Gamma,\tpvar{x}{w}{T},\Gamma' \vdash {J} &
%%         \oftp{\Gamma^w}{t}{T}}
\]

(Hopefully) admissible rules (with corresponding versions for terms):   
\[
\begin{array}{c}
\infer[\admiss]
      {\wfsub{\Gamma,\_}{\theta}{\Delta}}
      {\wfsub{\Gamma}{\theta}{\Delta}}
\qquad
\infer[\admiss]
      {\wfsub{{\Gamma}}{\idsubst}{\Gamma}}
      {}
\\ \\
\infer[\admiss]
      {\wfsub{{\Gamma_1}}{\tsubst{\theta_2}{\theta_1}}{\Gamma_3}}
      {\wfsub{{\Gamma_1}}{\theta_1}{\Gamma_2} &
        \wfsub{{\Gamma_2}}{\theta_2}{\Gamma_3} &
      }
\end{array}
\]

\subsection{Semantics}
Context extensions are interpreted as follows:
\begin{itemize}
	\item $\paren{\Gamma, \catvar x \co \C} \mapsto \Gamma \times \C$,
	\item $\paren{\Gamma, \catvar x \con \C} \mapsto \Gamma \times \op\C$,
	\item $\paren{\Gamma, \catvar u \coc T} \mapsto \int_\Gamma^+ T$,
	\item $\paren{\Gamma, \catvar u \conc T} \mapsto \int_\Gamma^- T$.
\end{itemize}

\section{Types}

\subsection{Universe}

The universe of sets is a directed type (e.g. the category of sets and
functions).

\[
\infer{\wfdtp{\types}}{}
\]

\FIXME{directed univalence}

\subsection{Op Categories}

\[
\begin{array}{c}
\infer{\wfdtp{\optp{\C}}}
      {\wfdtp{\C}}
\qquad
\infer{\ofdtp{\Gamma}{\optm{c}}{\optp{\C}}}
      {\ofdtp{\Gamma^\con}{c}{\C}}
\qquad
\infer{\ofdtp{\Gamma}{\unop{c}}{{\C}}}
      {\ofdtp{\Gamma^\con}{c}{\optp{\C}}}
\\ \\
\unop{\optm{c}} \deq c \\
\optm{\unop{c}} \deq c
\end{array}
\]

\subsection{Dagger Types}

We don't want to distinguish between sets and types. Dagger sets are not justifiable semantically and therefore should not exist.
%\[
%\begin{array}{c}
%\infer{\wftp{\Gamma}{\dagtp{T}}}
%      {\wftp{\Gamma^\con}{T}}
%\qquad
%\infer{\oftp{\Gamma}{\dagtm{t}}{\dagtp{T}}}
%      {\oftp{\Gamma^\con}{t}{T}}
%\qquad
%\infer{\oftp{\Gamma}{\undag{t}}{{T}}}
%      {\oftp{\Gamma^\con}{t}{\dagtp{T}}}
%\\ \\
%\undag{\dagtm{t}} \deq t \\
%\dagtm{\undag{t}} \deq t
%\end{array}
%\]

We can have presheaf dagger types! We would have
\begin{equation}
	{A^\dag(x)} = \forall(y, \vfi : x \to y).A(y) \,\text{naturally}.
\end{equation}
Then we get
\begin{align*}
	{A^{\dag \dag}(x)}
	&= \forall(y, \vfi : x \to y).A^\dag(y) \,\text{naturally} \\
	&= \forall(y, \vfi : x \to y, z, \psi : y \to z).A(z) \,\text{naturally} \\
	&\simeq \forall(y, \vfi : x \to y).A(y) \,\text{naturally} \\
	&\simeq A(x),
\end{align*}
twice using the Yoneda lemma. Including these in the theory seems as acceptable as including presheaf $\Pi$s.

You could even reduce functors $T : \Twisted\C \to \types$ to functors $T^\ddag : \C \to \types$, by setting
\begin{equation}
	T^\ddag(x) = \forall(y, \vfi : x \to y).T(\idm y) \,\text{naturally},
\end{equation}
where naturally means that if $t \in T^\ddag(x)$ and $x \xrightarrow{\vfi} y \xrightarrow{\psi} z$, then
\begin{equation}
	\xymatrix{
		t(y, \vfi) \in \ar@{|->}[rd] & T(\idm y) \ar[rd] \\
		& = & T(\psi) \\
		t(z, \psi \circ \vfi) \in \ar@{|->}[ru] & T(\idm z) \ar[ru]
	}
\end{equation}
Note that by the Yoneda lemma, $T^\ddag$ specializes to the identity operation for covariant $T$. For contravariant $T$, it specializes to $T^\dagger$. It is probably also precisely what is needed to define presheaf $\Pi$s if we have ordinary $\Pi$s.

\subsection{Hom Types}
\[
\begin{array}{c}
\infer{\ofdtp{\Gamma}{\Hom{\C}{c_0}{c_1}}{\types}}
      {\wfdtp{\C} &
        \oftp{\Gamma^\con}{c_0}{\C} & 
        \oftp{\Gamma}{c_1}{\C} 
      }
%\\ \\
%\infer{\oftpe{\Gamma}{\idm{c}}{\vcond}{\Hom{\C}{c^\vcond}{c}}}
%      {\oftpe{\dupctx{\Gamma}{\vcond}}{c}{\vcond}{\C} 
%      }
\end{array}
\]

The introduction rule is trivial/non-existent:
\[
	\infer{
		\oftp{\Gamma, \twcatvar{x}{\coc}{\C}}{\idm x}{\Hom{\C}{x^*}{x}}
	}{
		\wfctx{\Gamma} & \wfdtp{\C}
	}
\]

%% \infer{\oftpsf{\Delta}{\Gamma}{\morind{x_0,x_1,x.A}{M}{y.N}}{\subst{\subst{\subst{A}{M_0}{x_0}}{M_1}{x_1}}{M}{x}}}
%%       { \begin{array}{l}
%%           \wftpcc{\Delta}{\C} \\
%%           \wftpp{\Delta}{\Gamma,\vptptm{x_0}{\con}{\C},\vptptm{x_1}{\co}{\C},\vstptm{x}{\co}{\Hom{\C}{x_0}{x_1}}}{A} \\
%%           \oftpp{\Delta,\tptm{y}{\obtp{\C}}}{\Gamma}{N}{\subst{\subst{\subst{A}{\unob{y}}{x_0}}{\unob{y}}{x_1}}{\refl{y}}{x}} \\
%%           \oftpsf{\Delta}{\Gamma}{M}{\Hom{\C}{M_0}{M_1}} \\
%%         \end{array}
%%       }
%% \\ \\
%% \morind{}{\refl{M}}{y.N} \deq \subst{N}{M}{y}

Two-sided elimination is just substitution (and therefore admissible?):
\begin{equation}
	\infer[\admiss]{
		\oftp{\Gamma}{t[\substtw{c'}{c}{\vfi}{x}]}{S[\substtw{c'}{c}{\vfi}{x}]}
	}{ \begin{array}{l}
		\wfdtp{\C} \\
		\oftp{\Gamma^\con}{c'}{\C} \\
		\oftp{\Gamma}{c}{\C} \\
		\oftp{\Gamma}{\vfi}{\Hom{\C}{c'}{c}} \\
		\oftp{\Gamma, \twcatvar{x}{\co}{\C} }{S}{\types} \\
		\oftp{\Gamma, \twcatvar{x}{\co}{\C} }{t}{S}
	\end{array} }
\end{equation}

Source-based elimination, too, is just substitution (although here I'm not 100\% sure I stated what I wanted to state):
\begin{equation}
	\infer[\admiss]{
		\oftp{\Gamma}{t[c/x, \vfi/\xi]}{S[c/x, \vfi/\xi]}
	}{ \begin{array}{l}
		\wfdtp{\C} \\
		\oftp{\Gamma^\con}{c'}{\C} \\
		\oftp{\Gamma}{c}{\C} \\
		\oftp{\Gamma}{\vfi}{\Hom{\C}{c'}{c}} \\
		\oftp{\Gamma, \catvar{x}{\co}{\C}, \tpvar{\xi}{\coc}{\Hom{\C}{c'}{x}} }{S}{\types} \\
		\oftp{\Gamma, \catvar{x}{\co}{\C}, \tpvar{\xi}{\coc}{\Hom{\C}{c'}{x}} }{t}{S}
	\end{array} }
\end{equation}
\FIXME{contravariant derivable?}

\subsubsection{Try to prove hom composition}
There are several ways to even state it:

Corresponding to ordinary $\Pi$:
\[
	\infer{
		\oftp{\twcatvar x \co \C, \twcatvar y \co \C, \twcatvar z \co \C, \tpvar \vfi \conc{\Hom{\C}{x}{y^*}}, \tpvar \psi \conc{\Hom{\C}{y}{z^*}}}{}{\Hom{\C}{x^*}{z}}
	}{
		r
	}
\]

Corresponding to presheaf $\Pi$:
\[
	\infer{
		\oftp{\catvar x \con \C, \twcatvar y \co \C, \catvar z \co \C, \tpvar \vfi \coc{\Hom{\C}{x}{y}}, \tpvar \psi \coc{\Hom{\C}{y^*}{z}}}{}{\Hom{\C}{x}{z}}
	}{
		r
	}
\]

\subsubsection{Alternatives}
\paragraph{Introduction and elimination in the empty context.}
The following rules seem very sensible:
\[
	\infer{
		\oftp{\cdot}{\idm c}{\Hom{\C}{c}{c}}
	}{
		\oftp{\cdot}{c}{\C}
	}
\]
\begin{equation}
	\infer{
		\oftp{\cdot}{\ldots}{S[c/x, \vfi/\xi]}
	}{ \begin{array}{l}
		\wfdtp{\C} \\
		\oftp{\cdot}{c', c}{\C} \\
		\oftp{\cdot}{\vfi}{\Hom{\C}{c'}{c}} \\
		\oftp{\catvar{x}{\co}{\C}, \tpvar{\xi}{\coc}{\Hom{\C}{c'}{x}} }{S}{\types} \\
		\oftp{\cdot}{t}{S[c'/x, \idm{c'}/\xi]}
	\end{array} }
\end{equation}

\paragraph{Introduction and elimination in contexts that are somehow symmetric.}
If we naively unpack the old introduction rule, we obtain nonsense:
\[
	\infer{
		\oftp{\Gamma}{\idm c}{\Hom{\C}{c[\rho]}{c}}
	}{
		\oftp{\Gamma}{c}{\C} &
		\rho : \Gamma \cong \Gamma^\con
	}
\]
To see that this is nonsense, consider $\oftp{\catvar{x'}{\con}{\C}, \catvar x \co \C}{d}{\D}$. This would give us
\[
	\oftp{\catvar{x'}{\con}{\C}, \catvar x \co \C}{\idm d}{\Hom{\D}{d[x/x', x'/x]}{d}}
\]
but there is not in general a morphism from $d(x, x')$ to $d(x', x)$. There is one if we have a morphism from $x'$ to $x$, so we could have something like (this is the more precise unpacking of the old introduction rule):
\[
	\infer{
		\oftp{[\Gamma]}{\idm c}{\Hom{\C}{c[\Gamma/\Gamma', \Gamma^*/\Gamma]}{c[\Gamma^*/\Gamma']}}
	}{
		\oftp{\Gamma', \Gamma}{c}{\C} &
		\rho : \Gamma' \cong \Gamma^\con
	}
\]
We cannot have just $[\Gamma]$ up above, because then $c$ would be defined over $\Twisted{\Gamma}$ whereas the source of the morphism needs to be defined over $\op{\Twisted{\Gamma}}$.

\paragraph{Covariant Hom (Presheaf Hom).}
We could also consider a covariant (presheaf) Hom:
\[
	\infer{
		\ofdtp{\Gamma}{\Hombind{\C}{\coc}{c_0}{c_1}}{\types}
	}{
		\wfdtp{\C} &
		\oftp{\Gamma}{c_0}{\C} & 
		\oftp{\Gamma}{c_1}{\C} 
	}
\]
Semantically, $\Hombind{\C}{\coc}{c_0}{c_1}$ maps $\gamma \in \Gamma$ to the set of natural elements of
\[
	\forall (\gamma'' \in \Gamma). \forall (\phi : \gamma \to \gamma'').\Hom{\C}{c_0(\gamma'')}{c_1(\gamma'')}.
\]
This one doesn't interact well with boxed variables, but it admits a simple introduction rule:
\[
	\infer{
		\oftp{\Gamma}{\idm c}{\Hombind{\C}{\coc}{c}{c}}
	}{
		\oftp{\Gamma}{c}{\C}
	}
\]

\subsection{$\Pi$-like Things}

\subsubsection{Functor Categories}

\[
\begin{array}{c}
\infer{\wfdtp{\arr{\C_1}{\C_2}}}
      {\wfdtp{\C_1} &
        \wfdtp{\C_2}}
\qquad
\infer{\ofdtp{\Gamma}{\lam{x}{\C_1}{c}}{\arr {\C_1} {\C_2}}}
      {\ofdtp{\Gamma,\vptptm{x}{\co}{\C_1}}{c}{\C_2}}
\qquad
\infer{\ofdtp{\Gamma}{\app c {c'}}{\C_2}}
      {\ofdtp{\Gamma}{c}{\arr{\C_1}{\C_2}} &
        \ofdtp{\Gamma}{c'}{\C_1}}
\end{array}
\]

\[
\begin{array}{rcll}
\app {(\lam{x}{\C_1}{c})}{c'} & \equiv & \subst{c}{c'}{x} \\
\tptm{c}{\arr{\C_1}{\C_2}} & \equiv & {(\lam{x}{\C_1}{c}{x})} &
\end{array} \\
\]

\subsubsection{Covariant $\Pi$ (Presheaf $\Pi$)}
\begin{equation}
	\infer{
		\oftp{\Gamma}{\pico{x}{S}{T}}{\types}
	}{
		\oftp{\Gamma}{S}{\types} &
		\oftp{\Gamma, \tpvar x \coc S}{T}{\types}
	}
\end{equation}
\begin{equation}
	\infer{
		\oftp{\Gamma}{\laco x S t}{\pico{x}{S}{T}}
	}{
		\oftp{\Gamma, \tpvar x \coc S}{t}{T}
	}
	\qquad
	\infer{
		\oftp{\Gamma}{f s}{T[s/x]}
	}{
		\oftp{\Gamma}{f}{\pico{x}{S}{T}} &
		\oftp{\Gamma}{s}{S}
	}
\end{equation}

\[
\begin{array}{rcll}
\app {\laco x S t}{s} & \equiv & \subst{t}{s}{x} \\
\tpvar{f}{}{\pico{x}{S}{T}} & \equiv & \laco x S{fx} &
\end{array} \\
\]

The semantics of the presheaf $\Pi$ explicitly depend on the context, which makes the semantics of weakening/exchange/substitution a bit obscure.
\todoi{This obscurity could well be fatal!}

\subsubsection{Contravariant $\Pi$ (Ordinary $\Pi$)}
\begin{equation}
	\infer{
		\oftp{\Gamma}{\picon{x}{S}{T}}{\types}
	}{
		\oftp{\Gamma^\con}{S}{\types} &
		\oftp{\Gamma, \tpvar x \conc S}{T}{\types}
	}
\end{equation}
\begin{equation}
	\infer{
		\oftp{\Gamma}{\lacon x S t}{\picon{x}{S}{T}}
	}{
		\oftp{\Gamma, \tpvar x \conc S}{t}{T}
	}
	\qquad
	\infer{
		\oftp{\Gamma}{f s}{T[s/x]}
	}{
		\oftp{\Gamma}{f}{\picon{x}{S}{T}} &
		\oftp{\Gamma^\con}{s}{S}
	}
\end{equation}

\[
\begin{array}{rcll}
\app {\lacon x S t}{s} & \equiv & \subst{t}{s}{x} \\
\tpvar{f}{}{\picon{x}{S}{T}} & \equiv & \lacon x S{fx} &
\end{array} \\
\]

%\subsubsection{Dependent}
%
%
%\textbf{Allows dependency of $S_2$ on $S_1$, but whole type must be
%  covariant functor, not bifunctor, in intro and elim.}
%
%\[
%\infer{\oftp{\Gamma}{\piset{x}{S_1}{S_2}}{\types}}
%      {\oftp{\Gamma^\con}{S_1}{\types} &
%        \oftp{\Gamma, \tpvar {x}{\vcond}{S_1^\vcond}}{S_2}{\types}}
%\]
%Note that only the starred variables occur in $S_1^\vcond$ in the
%context, not any unstarred variables.  
%
%\[
%\begin{array}{c}
%\infer{\oftpe{\Gamma}{\ulam{x}{t}}{\bullet}{\piset x {S_1} {S_2}}}
%      { \ofdtp{\Gamma}{\piset{x}{S_1}{S_2}}{\types} & 
%        & \oftpe{\Gamma, \tpvar x {\vcond} {S_1^\vcond} }{t}{\bullet}{S_2}}
%\qquad
%\infer{\oftpe{\Gamma}{\app t {t'}}{\bullet}{\tsubst {S_2} {{t'}/ x}}}
%      {\oftpe{\Gamma}{t}{\bullet}{\piset{x}{S_1}{S_2}} &
%        \oftpe{\Gamma}{t'}{\vcond}{S_1^\vcond}}
%\end{array}
%\]
%Note that these rules only apply when the type is well-formed in
%$\Gamma$, not with the secret context, which we notate by writing
%$\bullet$ for a fresh variable constructor variable that doesn't occur
%anywhere.
%
%\[
%\begin{array}{rcll}
%\app {(\ulam{x}{t})}{t'} & \equiv & \subst{t}{t'}{x} \\
%\tpvar{t}{\bullet}{\piset{x}{S_1}{S_2}} & \equiv & {(\ulam{x}{t}{x})} &
%\end{array} \\
%\]
%
%\subsubsection{Non-dependent functions between sets}
%
%\textbf{Works for a bifunctor, but no dependency of $S_2$ on $S_1$.}
%
%\[
%\infer{\oftp{\Gamma}{\arr{S_1}{S_2}}{\types}}
%      {\oftp{\Gamma^\con}{S_1}{\types} &
%        \oftp{\Gamma}{S_2}{\types}}
%\]
%Note that the starred variables are not free in $S_1$ because we wrote
%$\Gamma^\con$ not $\dupctx{\Gamma^\con}{\vcond}$ in the premise, so $S_1$ is
%weakened in the context extension.  This is because we want to take the
%$\Pi$ of an opfibration and a fibration, not a bifibration and a
%fibration, or something?  Semantically, the context extension will mean
%the co/contravariant category of elements instead of the end.  
%
%\[
%\begin{array}{c}
%\infer{\oftpe{\Gamma}{\ulam{x}{t}}{\vcond}{\arr {S_1} {S_2}}}
%      {\oftpe{\Gamma, \tpvar x {\vcond} {S_1^\vcond} }{t}{\vcond}{S_2}}
%\qquad
%\infer{\oftpe{\Gamma}{\app t {t'}}{\vcond}{{S_2}}}
%      {\oftpe{\Gamma}{t}{\vcond}{\arr{S_1}{S_2}} &
%        \oftpe{\Gamma}{t'}{\vcond}{S_1^\vcond}}
%\end{array}
%\]
%Note that these rules only apply when the type is well-formed in
%$\Gamma$, not with the secret context, which we notate by writing
%$\bullet$ for a fresh variable constructor variable that doesn't occur
%anywhere.
%
%\[
%\begin{array}{rcll}
%\app {(\ulam{x}{t})}{t'} & \equiv & \subst{t}{t'}{x} \\
%\tptm{t}{\arr{S_1}{S_2}} & \equiv & {(\ulam{x}{t}{x})} &
%\end{array} \\
%\]

\subsubsection{$\Pi$ of a category and a set (Limit)}
\[
\begin{array}{c}
\infer{\ofdtp{\Gamma}{\picat{x}{\C}{S}}{\types}}
      {\wfdtp{\C} &
        \ofdtp{\Gamma,\catvar{x}{\co}{\C}}{S}{\types}}
\qquad
\infer{\oftp{\Gamma}{\lam{z}{\C}{t}}{\picat {x} \C S}}
      {\oftp{\Gamma,\catvar{z}{\co}{\C}}{t}{\tsubst{S}{z/x}}}
\qquad
\infer{\oftp{\Gamma}{\app f {c}}{\tsubst S {c/x}}}
      {\oftp{\Gamma}{f}{\picat{x}{\C}{S}} &
        \ofdtp{\Gamma}{c}{\C}}
\end{array}
\]

\[
\begin{array}{rcll}
\app {(\lam{z}{\C}{t})}{c} & \equiv & \subst{t}{c}{z} \\
\tptm{f}{\picat{(x,y)}{\C}{S}} & \equiv & {(\lam{z}{\C}{f}{z})} &
\end{array} \\
\]
The contravariant version should be admissible using opposites.

\subsection{$\Sigma$-like things}

\subsubsection{Product Categories}
Directed types have simple functions and products (exponentials and
products in \Cat).


\[
\begin{array}{c}
\infer{\wfdtp{\prd{\C_1}{\C_2}}}
      {\wfdtp{\C_1} &
        \wfdtp{\C_2}}
\qquad
\infer{\ofdtp{\Gamma}{\pair{c_1}{c_2}}{\prd {\C_1} {\C_2}}}
      {\ofdtp{\Gamma}{c_1}{\C_1} &
       \ofdtp{\Gamma}{c_2}{\C_2}}
\\ \\
\infer{\ofdtp{\Gamma}{\fst c}{\C_1}}
      {\ofdtp{\Gamma}{c}{\prd{\C_1}{\C_2}}}
\qquad
\infer{\ofdtp{\Gamma}{\snd c}{\C_2}}
      {\ofdtp{\Gamma}{c}{\prd{\C_1}{\C_2}}}
\end{array}
\]

\[
\begin{array}{rcll}
\fst{\pair{c_1}{c_2}} & \deq & c_1 \\
\snd{\pair{c_1}{c_2}} & \deq & c_2 \\
\tptm{c}{\prd{\C_1}{\C_2}} & \deq & \pair{\fst c}{\snd c}
\end{array}
\]

\subsubsection{Category of elements}
\[
\begin{array}{c}
\infer{\wfdtp{\sigmacat{x}{\C}{S}}}
      {\wfdtp{\C} &
        \wftp{\catvar{x}{\co}{\C}}{S}}
\qquad
\infer{\ofdtp{\Gamma}{\pair{c}{t}}{\sigmacat x \C S}}
      {\ofdtp{\Gamma}{c}{\C} &
       \oftp{\Gamma}{t}{\tsubst{S}{c/x}}}
\\ \\
\infer{\ofdtp{\Gamma}{\fst c}{\C}}
      {\ofdtp{\Gamma}{c}{\sigmacat{x}{\C}{S}}}
\qquad
\infer{\oftp{\Gamma}{\snd c}{\tsubst{S}{c/x}}}
      {\ofdtp{\Gamma}{c}{\sigmacat{x}{\C}{S}}}
\end{array}
\]
The contravariant version should be admissible using opposites.

\subsubsection{Category of elements of a twisted-arrow-functor ($\hat{\Sigma}$)}
A section of the covariant or contravariant Grothendieck construction (category of elements) is an element of the limit. For bifunctors $T: \op\C \times \C \to \types$, we can define a category of elements whose sections are ends. All of these have a common generalization: from a functor $S : \Twisted{\C} \to \types$ on the twisted arrow category of $\C$, we can make a category $\widetilde\Sigma_\C S$ with a projection to $\C$, whose sections are, curiously, in correspondence with elements of the limit of $S$, even though they are functors on $\C$, unlike $S$ itself.

\todoi{Projection and $\beta$-rules have not been updated. Here, we need to know how we handle real identity morphisms, on terms rather than on boxed variables.}
\[
\begin{array}{c}
\infer{\wfdtp{\sigmatwcat{x}{\C}{S}}}
      {\wfdtp{\C} &
        \wftp{\twcatvar{x}{\co}{\C}}{S}}
\qquad
\infer{\ofdtp{\cdot}{\pair{c}{t}}{\sigmatwcat x \C S}}
      {\ofdtp{\cdot}{c}{\C} &
       \oftp{\cdot}{t}{S[\substtw{c}{c}{\idm c}{x}]}}
\\ \\
\infer{\ofdtp{\Gamma}{\fst c}{\C}}
      {\ofdtp{\Gamma}{c}{\sigmacatstar{\vcond}{x}{\C}{S}}}
\qquad
\infer{\oftpe{\Gamma}{\snd c}{\vcond}{\tsubst{S}{\substdup{(\fst c^\vcond / x)}{\vcond}{\fst c/x}}}}
      {\ofdtp{\Gamma}{c}{\sigmacatstar{\vcond}{x}{\C}{T}}}
\end{array}
\]

\[
\begin{array}{rcll}
\fst{\pair{c}{t}} & \deq & c \\
\snd{\pair{c}{t}} & \deq & t \\
\tptm{c}{\sigmacl{x}{\C}{S}} & \deq & \pair{\fst c}{\snd c}
\end{array}
\]

\subsubsection{$\Sigma$ of sets}

\[
\begin{array}{c}
\infer{\ofdtp{\Gamma}{\sigmaset{x}{S}{T}}{\types}}
      {\ofdtp{\Gamma}{S}{\types} &
        \ofdtp{\Gamma, \tpvar{x}{\coc}{S}}{T}{\types}}
\qquad
\infer{\oftp{\Gamma}{\pair{s}{t}}{\sigmaset x {S} {T}}}
      {\ofdtp{\Gamma}{S}{\types} &
        \ofdtp{\Gamma, \tpvar{x}{\coc}{S}}{T}{\types} &
        \oftp{\Gamma}{s}{S} &
        \oftp{\Gamma}{t}{\subst{T}{s}{x}}}
\\ \\
\infer{\oftp{\Gamma}{\fst t}{S}}
      {\oftp{\Gamma}{t}{\sigmaset{x}{S}{T}}}
\qquad
\infer{\oftp{\Gamma}{\snd t}{\subst{T}{\fst t}{x}}}
      {\oftp{\Gamma}{t}{\sigmaset{x}{S}{T}}}
\end{array}
\]

\[
\begin{array}{rcll}
\fst{\pair{s}{t}} & \deq & s \\
\snd{\pair{s}{t}} & \deq & t \\
\tpvar{t}{}{\sigmaset{x}{S_1}{S_2}} & \deq & \pair{\fst t}{\snd t}
\end{array}
\]

%\subsubsection{Non-dependent product of sets}
%
%\[
%\begin{array}{c}
%\infer{\ofdtp{\Gamma}{\prd{S_1}{S_1}}{\types}}
%      {\ofdtp{\Gamma}{S_1}{\types} &
%        \ofdtp{\Gamma}{S_2}{\types}}
%\qquad
%\infer{\oftpe{\Gamma}{\pair{t_1}{t_2}}{\vcond}{\prd {S_1} {S_2}}}
%      { \oftpe{\Gamma}{t_1}{\vcond}{S_1} &
%        \oftpe{\Gamma}{t_2}{\vcond}{S_2}}
%\\ \\
%\infer{\oftpe{\Gamma}{\fst t}{\vcond}{{S_1}}}
%      {\oftpe{\Gamma}{t}{\vcond}{\prd{S_1}{S_2}}}
%\qquad
%\infer{\oftpe{\Gamma}{\snd t}{\vcond}{{S_2}}}
%      {\oftpe{\Gamma}{t}{\vcond}{\prd{S_1}{S_2}}}
%\end{array}
%\]
%
%\[
%\begin{array}{rcll}
%\fst{\pair{s}{t}} & \deq & s \\
%\snd{\pair{s}{t}} & \deq & t \\
%\tptm{t}{\prd{S_1}{S_2}} & \deq & \pair{\fst t}{\snd t}
%\end{array}
%\]

\subsection{Identity types}
\[
	\infer{
		\ofdtp{\Gamma}{\idtp A a b}{\types}
	}{
		%\ofdtp{\Gamma}{A}{\types} &
		\oftp{\Gamma}{a, b}{A}
	}
	\qquad
	\infer{
		\oftp{\Gamma}{\refl\,a}{\idtp A a a}
	}{
		\oftp{\Gamma}{a}{A}
	}
	\qquad
	\infer{
		\oftp{\Gamma}{\ldots}{S[a/x,b/y,q/p]}
	}{
		\begin{array}{l}
			\ofdtp{\Gamma}{A}{\types} \\
			\ofdtp{\Gamma, \tpvar{x}{\coc}{A}, \tpvar{y}{\coc}{A}, \tpvar{p}{\coc}{\idtp A x y}}{S}{\types} \\
			\oftp{\Gamma}{a, b}{A} \\
			\oftp{\Gamma}{s}{S[a/x, a/y, \refl\,a/p]} \\
			\oftp{\Gamma}{q}{\idtp A a b}
		\end{array}
	}
\]

\todoi{What is below, is copied from an earlier version and NOT updated.}
\subsection{Bool}

\[
\begin{array}{c}
\infer{\ofdtp{\Gamma}{\bool}{\types}}
      {}
\qquad
\infer{\oftpe{\Gamma}{\dsd{true}}{\bullet}{\bool}}{}
\qquad
\infer{\oftpe{\Gamma}{\dsd{false}}{\bullet}{\bool}}{}
\\ \\
\infer{\oftpe{\Gamma}{\ite{\tptm{x}{\dsd{2}}.S}{t_1}{t_2}{t}}{\vcond}{\tsubst{S}{\substdup{(t^\vcond/x)}{\vcond}{t/x}}}}
      {\ofdtp{\dupctx{\Gamma,\tpvar{x}{\bullet}{\dsd{2}}}{\vcond}}{S}{\types} &
        \oftpe{\Gamma}{t}{\bullet}{\dsd{2}} &
        \oftpe{\Gamma}{t_1}{\vcond}{\tsubst{S}{\substdup{(\dsd{true}^\vcond/x)}{\vcond}{\dsd{true}/x}}} &
        \oftpe{\Gamma}{t_2}{\vcond}{\tsubst{S}{\substdup{(\dsd{false}^\vcond/x)}{\vcond}{\dsd{false}/x}}}
      }
\\ \\
\infer{\oftp{\Gamma}{\ite{}{t_1}{t_2}{t}}{\C}}
      { \oftpe{\Gamma}{t}{\bullet}{\dsd{2}} &
        \ofdtp{\Gamma}{t_1}{\C} &
        \ofdtp{\Gamma}{t_2}{\C}
      }
\\ \\
\begin{array}{rcll}
\ite{}{M_1}{M_2}{\dsd{true}} & \equiv & M_1 \\
\ite{}{M_1}{M_2}{\dsd{false}} & \equiv & M_2 \\
\end{array} \\
\end{array}
\]

\subsection{Interval}

\[
\begin{array}{c}
\infer{\wftp{\Gamma}{\I}}
      {}
\qquad
\infer{\oftp{\Gamma}{\Iz}{\I}}{}
\qquad
\infer{\oftp{\Gamma}{\Io}{\I}}{}
%% \infer{\oftp{\Gamma}{\ite{\vstptm{x}{v}{\dsd{2}}.A}{N_1}{N_2}{t}}{\subst{A}{t}{x}}}
%%       {\wftp{\Gamma,\vstptm{x}{v}{\dsd{2}}}{A} &
%%         \oftp{\Gamma^v}{M}{\dsd{2}} &
%%         \oftp{\Gamma}{N_1}{\subst{A}{\dsd{true}}{x}} &
%%         \oftp{\Gamma}{N_2}{\subst{A}{\dsd{false}}{x}}
%%       }
%% \\ \\
%% \begin{array}{rcll}
%% \ite{}{M_1}{M_2}{\dsd{true}} & \equiv & M_1 \\
%% \ite{}{M_1}{M_2}{\dsd{false}} & \equiv & M_2 \\
%%\end{array} 
\end{array}
\]

\FIXME{elim}

\subsection{Equality of Elements of Types}

%% This could be done like ITT instead.  

%% \[
%% \begin{array}{c}
%% \infer{\wftpsf{\Delta}{\Gamma}{\Eq{S}{M}{N}}}
%%       {\wftpsf{\Delta}{\Gamma}{S} &
%%         \oftpsf{\Delta}{\Gamma}{M}{S} & 
%%         \oftpsf{\Delta}{\Gamma}{N}{S} 
%%       }
%% \qquad
%% \infer{\oftpsf{\Delta}{\Gamma}{\refl{M}}{\Eq{S}{M}{M}}}
%%       {\oftpsf{\Delta}{\Gamma}{M}{S}}
%% \\ \\
%% \infer{M \deq N}
%%       {\oftpsf{\Delta}{\Gamma}{P}{\Eq{S}{M}{N}}}
%% \qquad
%% \infer{P \deq \refl{M}}
%%       {\oftpsf{\Delta}{\Gamma}{P}{\Eq{S}{M}{N}}}
%% \end{array}
%% \]

\section{Examples}

The contravariant sum is derivable:
\begin{equation}
	\infer{
		\wftp{ \Gamma }{ \paren{\sigmatp{x}{S}{(T^\dagger)}}^\dagger }
	}{
		\infer{
			\wftp{ \Gamma^- }{ \sigmatp{x}{S}{(T^\dagger)} }
		}{
			\ofdtp{ \Gamma^- }{ S }{ \types }
			\qquad \qquad
			\infer{
				\wftp{\Gamma^-, \tpvar{x}{\coc}{\Elp S}}{T^\dagger}
			}{
				\wftp{\Gamma^+, \tpvar{x}{\conc}{\Elp S}}{T}
			}
		}	
	}
\end{equation}
\begin{equation}
	\infer{
		\oftp{\Gamma}{\dagtm{(\undag s, \dagtm t)}}{ \paren{\sigmatp{x}{S}{(T^\dagger)}}^\dagger }
	}{
		\oftp{\Gamma}{s}{\Elp S^\dagger}
		\qquad \qquad
		\oftp{\Gamma}{t}{\subst T {\undag s} x}
	}
\end{equation}
We have projections $\dagtm{\fst{\undag -}}$ and $\undag{\snd{\undag -}}$.

\paragraph{Composition}

%% Suppose $\wftpcc{\Delta}{\C}$, then we have 

%% \[
%% \begin{array}{rcl}
%% comp & : & \piinv{x}{\obtp{\C}}{\piinv{y}{\obtp{\C}}{\piinv{z}{\obtp{\C}}{}}}\\
%%      &   & {\piinv{p}{\Hom{\C}{\unob{x}}{\unob{y}}}{\piinv{q}{\Hom{\C}{\unob{y}}{\unob{z}}}{\Hom{\C}{\unob{x}}{\unob{z}}}}}
%% \end{array}
%% \]


\section{Properties}

\section{Examples}

%% \subsection{\op{-} is an involution}

%% Suppose \wftpc{\Delta}{\Gamma^w}{A}.  

%% \[
%% \infer{\voftp{\Gamma,\catvar{x}{\co}{w}{A}}{\optm{\optm{x}}}{w}{\optp{(\optp{A})}}}
%%       {\infer{\voftp{\Gamma^\con,\catvar{x}{\con}{\flip{w}}{A}}{\optm{x}}{\flip{w}}{{\optp{A}}}}
%%              {{\voftp{\Gamma,\catvar{x}{\co}{{w}}{A}}{x}{{w}}{{{A}}}}}}
%% \]

%% \[
%% \infer{\voftp{\Gamma,\catvar{x}{\co}{w}{\optp{(\optp{A})}}}{\letop{}{x}{y.{\letop{}{y}{z.z}}}}{w}{A}}
%%       {{\vvoftp{\Gamma,\catvar{x}{\co}{w}{\optp{(\optp{A})}}}{x}{\co}{w}{\optp{(\optp{A})}}} &
%%         \infer{\voftp{\Gamma,\catvar{x}{\co}{w}{\optp{(\optp{A})}},\catvar{y}{\con}{w}{\op{A}}}{\letop{}{y}{z.z}}{w}{A}}
%%               {\begin{array}{l}
%%                   \vvoftp{\Gamma,\catvar{x}{\co}{w}{\optp{(\optp{A})}},\catvar{y}{\con}{w}{\op{A}}}{y}{\con}{w}{\op{A}} \\
%%                   \vvoftp{\Gamma,\catvar{x}{\co}{w}{\optp{(\optp{A})}},\catvar{y}{\con}{w}{\op{A}},\catvar{z}{\co}{w}{A}}{z}{\co}{w}{A}
%%                 \end{array}}
%%       }
%% \]

%% \[
%% {\Gamma,\catvar{x}{\co}{w}{A}} \vdash {\letop{}{{\optm{\optm{x}}}}{y.{\letop{}{y}{z.z}}}} \deq x
%% \]

%% \FIXME{need some sort of identity type to do the other direction}

%% %% \[
%% %% {\Gamma,\catvar{x}{\co}{w}{\op{(\op{A})}} \vdash {\letop{}{x}{y.{\letop{}{y}{z.z}}}} \deq x
%% %% \]

%% \subsection{Paths in \op{-}}

%% Suppose \wftpc{\Delta}{\Gamma^w}{A} and \voftp{\Gamma,\vdimtm{x}{\co}}{a}{w}{A}.

%% \[
%% \infer{\voftp{\Gamma,\vdimtm{x}{\co}}{\optm{?}}{w}{\op{A}}}
%%       {\voftp{{\Gamma}^\con,\vdimtm{x}{\con}}{?}{\flip{w}}{A}}
%% \]



%% \subsection{Core and Op}

%% \begin{enumerate}
%% \item $A$ and $\op{(\op{A})}$. Suppose \wftp{\Gamma^w}{A}.  

%% If \voftp{\Gamma}{M}{w}{A}, then we have
%% \[
%% \infer{\voftp{\Gamma}{\optm{\optm{M}}}{w}{\op{(\op{A})}}}
%%       {\infer{\voftp{\Gamma^\con}{\optm{M}}{\flip{w}}{\op{A}}}
%%              {\voftp{{\Gamma^\con}^\con}{M}{\flip{\flip{w}}}{A}}}
%% \]
%% using ${{\Gamma^\con}^\con} = \Gamma$ 
%% and ${\flip{\flip{w}}} = w$.  

%% If \voftp{\Gamma}{M}{w}{\op{(\op{A})}} then we have 
%% \[
%% \infer{\voftp{\Gamma}{\letop{}{\co}{x.M}{\letop{}{\con}{x}{y.y}}}{w}{A}}
%%       {{\voftp{\Gamma}{M}{w}{\op{(\op{A})}}} &
%%         \infer{\voftp{\ectx{\Gamma}{x}{\con}{w}{\op{A}}}{\letop{}{\con}{x}{y.y}}{w}{A}}
%%               {\infer{\vvoftp{\ectx{\Gamma}{x}{\co}{w}{\op{A}}}{x}{\con}{w}{\op{A}}}
%%                      {\infer{\voftp{\ectx{\Gamma^\con}{x}{\co}{\flip{w}}{\op{A}}}{x}{\flip{w}}{\op{A}}}{}} &
%%                \infer{\voftp{\ectx{\ectx{\Gamma}{x}{\con}{w}{\op{A}}}{y}{\co}{w}{A}}{y}{w}{A}}
%%                      {}
%%               }}
%% \]

%% \item \Core{(\Core{A})} and \Core{A}.  Suppose \wftp{\Gamma^w}{A}.  

%% If \voftp{\Gamma}{M}{w}{\Core{A}}, then 
%% \[
%% \infer{\voftp{\Gamma}{?}{w}{\Core{A}}}
%%       {\voftp{\Gamma}{M}{w}{\Core{A}} &
%%         \voftp{\ectx{\Gamma}{x}{\inv}{w}{A}}{?}{w}{\Core{(\Core{A})}}
%%       }
%% \]

%% \end{enumerate}

%% \paragraph{Symmetry}

%% Suppose \wftp{\Gamma}{A}.
%% The following context is well-formed (for any $d$):
%% \[
%% \Gamma_0 := \coctx {\coctx {\coctx \Gamma x A \backd} y A \ford} p {\Hom{\wtp{A}{\ford}}{x}{y}} d
%% \]
%% However, the judgement 
%% \[
%% \wftp{\Gamma_0}{\Hom{\wtp{A}{\ford}} y x}
%% \]
%% is not derivable, because to show this, we would need that
%% $\wftp{\Gamma_0}{A}$ (which is true, by weakening) and 
%% \oftp{\Gamma_0}{y}{A}{\backd} 
%% and 
%% \oftp{\Gamma_0}{x}{A}{\ford}.  These are not derivable, 
%% it would require, e.g. $\wtptm{y}{d}{A} \in \Gamma_0$ for $d \le \backd$,
%% but we have $\wtptm{y}{\ford}{A} \in \Gamma_0$, and $\ford \not \le
%% \backd$.  

%% Dually, in 
%% \[
%% \Gamma_1 := \coctx {\coctx {\coctx \Gamma x A \ford} y A \backd} p {\Hom{\wtp{A}{\backd}}{x}{y}} d
%% \]
%% it is not the case that $\wftp{\Gamma_1}{\Hom{\wtp{A}{\backd}}{y}{x}}$.  

%% However, in 
%% \[
%% \Gamma_3 := \coctx {\coctx {\coctx \Gamma x A \symd} y A \symd} p {\Hom{\wtp{A}{\symd}}{x}{y}} d
%% \]
%% we have \wftp{\Gamma_3}{\Hom{\wtp{A}{\symd}}{y}{x}} because
%% \wftp{\Gamma_3}{A} (by weakening) 
%% and \woftp{\Gamma_3}{y}{A}{\invd{\symd}}
%% and \woftp{\Gamma_3}{x}{A}{\symd}.  Thus, we can apply $\dsd{J}$, and to
%% derive
%% \[
%% \woftp{\Gamma_3}{\jay{\Gamma_3.{\Hom{\wtp{A}{\symd}}{y}{x}}}{?}{p}}{\Hom{\wtp{A}{\symd}}{y}{x}}{d}
%% \]
%% it suffices to derive
%% \[
%% \oftp{\coctx \Gamma x A \symd}{\refl{x}}{\Hom{\wtp{A}{\symd}}{x}{x}}
%% \]
%% which is true.  

%% Thus, symmetry is not statable when $A$ is considered with direction
%% \ford\/ or \backd, but true when considered at direction \symd.  

%% \paragraph{Groupoid Structure}

%% Assume \wftp{\Gamma}{A}.

%% The following context is well-formed (for any $d_1,d_2$):

%% \[
%% \Gamma_0 := \coctx {\coctx {\coctx {\coctx {\coctx \Gamma x A \backd} y A \symd} z A \ford} p {\Hom{\wtp{A}{\ford}}{x}{y}} {d_1}} q {\Hom{\wtp{A}{\ford}}{y}{z}} {d_2}
%% \]
%% Note that $y$ must be symmetric, because it occurs on both sides of an equation.  

%% \wftp{\Gamma_0}{\Hom{\wtp{A}{\ford}}{x}{z}} because 
%% \woftp{\Gamma_0}{x}{A}{\backd} and 
%% \woftp{\Gamma_0}{z}{A}{\ford}.

%% \paragraph{Tethering}

%% The elimination rule for \dsd{bool} is ``untethered'': the direction of
%% the boolean being analyzed is unrelated to the direction of the 
%% result.  This allows you to define, for example,

%% \[
%% \woftp{\coctx \cdot x {\dsd{2}} \ford}{\ite{\_.\dsd{2}}{\dsd{true}}{\dsd{false}}{x}}{\dsd{2}}{\backd}
%% \]
%% which casts a forward boolean to a backwards one.  The opposite is also
%% definable, and (conjecture) these should give an equivalence between
%% \dsd{2} and \op{\dsd{2}} (use bool-elim to prove that they compose to
%% the identity).  

%% Similarly, the elimination rule for $\Sigma$ is untethered; for example,
%% for \wftp{\Gamma}{A},
%% \[
%% \woftp{\coctx \Gamma p {\wsigma x \ford A B}
%%   \ford}{\splitsig{\_.\wsigma{x}{\backd}{A}{B}}{x,y.(\optm x, \optm y)}{p}}{\wsigma x \backd
%%   {\op A} {\op B}}{\backd}
%% \]
%% So we get a (covariant) map from $\wsigma x \ford A B$ to $\op{(\wsigma
%%   x \backd {\op A} {\op B})}$, which (conjecture) should be an
%% equivalence.    

%% \paragraph{Transport}

%% \paragraph{Ap}

%% \paragraph{Properties of Op}

%% \paragraph{Swapping directions}

%% \subsection{Exact Equalities}

%% Add new world $\dsd{exact} \le \symd$.  Variations:
%% \begin{itemize}
%% \item no extra axioms: usual computational interp with $J$ just
%%   computing on \refls\/ should work if you only ever use exact equalities.

%% \item extensional: add equality reflection for 
%% \Hom{A^\dsd{exact}}{M}{N}.

%% \item OTT-style: add funext and
%% propositional univalence?  
%% \end{itemize}
%% In all of these, hopefully \Hom{A^\dsd{exact}}{M}{N} and any transports
%% on it are erasable at run-time.  

%% \subsection{Open Issues}

%% \begin{enumerate}
%% \item Check that the typing makes sense for the defeq rules; should
%%   both terms have the same direction $d$?

%% \item \wftp{\op \Gamma}{A} for $\Pi$ and \wftp{\Gamma}{A} for \dsd{Hom}
%%   are necessary, right?

%% \item Do you need the labels on $\Pi$ and $\Sigma$?  Or can types be
%%   more modal, with the direction being passed in from the outside?  

%%   Problem: need to write \wtptm{x}{d}{A} and \wtptm{x}{\invd d}{A} for
%%   the same $A$, so you don't want $d$ to need to be something specific.

%% \item Would coprodcuts/lists be for a labelled type, or would you flow
%%   the direction in from outside?

%% \item What do the directions on
%% \wtptm{p}{d}{\Hom{\wtp{A}{\ford}}{M}{N}} mean?
%% E.g. the ? in $\Ielimfor{}{}{}{}{}$.

%% \item Should $d_1$ and $d_2$ in J be the same?

%% \end{enumerate}

\section{Related Work}

\subsection{Warren's Directed Type Theory}

\begin{verbatim}
x',x,y,y',u: \Hom{x'}{x}, v:\Hom{x}{y}, w: \Hom{y}{y'}, Delta |- E
x,y,u,Delta |- E[x/x',y/y',refl/u,refl/v]
------------------------------------------
x',x,y,y',u: \Hom{x'}{x}, v:\Hom{x}{y}, w: \Hom{y}{y'}, Delta |- J(...) : E

If z\in{x',x,y,y'} occurs in Delta, then it occurs in E
x and y occur in E[x/x',y/y',refl/u,refl/v]

E.g. 

x',x,y,y',u,v,w |- Hom(y,x) -- y and x don't get contracted
x',x,y,y',u,v,w |- Hom(x,x') -- doesn't mention y

x',x,y,y',u,v,w |- Hom(x,x') * Hom(x,y) -- need to extend the side conditions for pairs...
\end{verbatim}


%% \setlength{\bibsep}{-1pt} %% dirty trick: make this negative
{ %% \small
%% \linespread{0.70}
\bibliographystyle{abbrvnat}
\bibliography{drl-common/cs}
}


\end{document}
